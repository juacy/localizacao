\documentclass[10pt,a4paper]{article}
\usepackage[utf8]{inputenc}
\usepackage[T1]{fontenc}
\usepackage{amsmath}
\usepackage{amssymb}
\usepackage{amsthm}
\usepackage{graphicx}
\usepackage{cancel}

\newcommand\ds{\displaystyle}

\newtheorem{theorem}{Teorema}[section]
\newtheorem{lemma}[theorem]{Teorema}
\newtheorem{corollary}[theorem]{Corolário}
\newtheorem{proposition}[theorem]{Proposição}
\newtheorem{example}[theorem]{Exemplo}
\newtheorem{remark}[theorem]{Observação}
\newtheorem{notation}[theorem]{Notação}
\newtheorem{definition}[theorem]{Definição}
\newtheorem{nota}[theorem]{Remark}

\begin{document}
\title {Estudos TCC}
\maketitle
\vspace{0.7cm}

\section{Anéis}
 
\definition[Anel]{Um anel R é um conjunto com duas operações $+ \textrm{ e } *$ tal que dados $x, y, z \in R$ temos:
\begin{enumerate}
	\item $x+(y+z)=(x+y)+z$
	\item $x+y \in R$
	\item $\exists 0$ tal que $\forall x, x+0=x$
	\item $x+y=y+x$
	\item $\exists -x$ tal que $ x, x+(-x)=x$
	\item $a*b \in R$
	\item $(a*b)*c=a*(b*c)$
	\item $a*(b+c)=a*b+a*c$
	\item $\exists 1 \textrm{ tal que } a*1=1*a=a$
\end{enumerate}
}

\remark{A definição de anel varia de autor para autor, alguns consideram anéis comutativos com unidade, outro já não consideram a existência do neutro multiplicativo}

Quando a multiplicação também é comutativa, isto é $a*b=b*a$ chamamos de anel comutativo.

\example{Temos que $\mathbb{Z},\mathbb{Q}, \mathbb{R} $ com as operações usuais de soma e produto são anéis.}

\proposition{Seja R um anel não trivial, se $x \in R$ então $x*0=0$}
\begin{proof}
Temos que $$x*0=x*(0+0)=x*0+x*0$$ mas como existe o oposto de $x*0$ podemos somar
$$x*0-x*0=x*0+x*0-x*0$$ logo $$0=x*0$$
\end{proof}

Temos que caso $1=0$ o anel é trivial. Já que $x=x*1=x*0=0$.

\definition[Subaneis]

\definition[Homomorfismo] Uma função entre os anéis $A$ e $B$ é dito homomorfismo se:
\begin{enumerate}
	\item $f(x+y)=f(x)+f(y)$
	\item $f(xy)=f(x)f(y)$
\end{enumerate}
 
\definition[Ideais(maximal e primo)]

\definition[Divisores de zero]

\definition[Anéis quociente]


\section{Corpo de fração}

\section{Localização em anéis comutativos}

Vamos seguir a demonstração a partir de um anel $A$ sem unidade, no livro temos a demonstração feita com anel com unidade.

\begin{definition}
	Um subconjunto S de um anel A é dito um conjunto multiplicativo se $1 \in S$ e $ x\cdot y \in S$ para todo $x , y \in S$.
\end{definition}

Agora que já temos todas as definições necessárias para iniciar a construção, vamos começar.

\begin{definition}
	Seja A um anel e S um conjunto multiplicativo de A. Vamos definir uma relação em $A \times S$ como $$(a,b) \equiv (c, d) \Leftrightarrow (ad-bc)u=0$$ para algum u $\in S$.
\end{definition}

Esta relação, é uma relação de equivalência.

\begin{proof} Vamos mostrar que é uma relação reflexiva, simétrica e transitiva.
	
	\begin{enumerate}
		\item [Reflexiva]  $(a,b) \equiv (a, b)$, de fato, pois $ab-ba = 0$, já que estamos trabalhando com um anel comutativo.
		\item [Simétrica] Temos que $(a,b) \equiv (c, d)$ nos leva a $(ad-bc)u=0$ mas como o anel é comutativo podemos trocar a ordem dos fatores $(da-cb)u=0$. Multiplicando por -1 chegamos a $(cb-da)u=0$ que é igual a $(c,d) \equiv (a, b)$.
		\item [Transitividade] Seja $(a,b) \equiv (c, d)$ e $(c,d) \equiv (e, f)$, devemos chegar em $(a,b) \equiv (e, f)$.
		De $(a,b) \equiv (c, d)$ temos $(ad-bc)u_1=0$ para algum $u_1 \in S$.
		
		De $(c,d) \equiv (e, f)$ temos $(cf-de)u_2=0$ para algum $u_2 \in S$.
		
		Multiplicando a primeira equação por $fu_2$ e a segunda por $bu_1$ chegamos a 
		$$(fad-fbc)u_2u_1=0$$
		$$(bcf-bde)u_2u_1=0$$
		
		Novamente, vamos usar a comutatividade para reorganizar as expressões
		
		$$(afd-fbc)u_2u_1=0 \ (*)$$
		$$(fbc-bed)u_2u_1=0$$
		
		Da segunda expressão temos que $fbcu_2u_1 = bedu_2u_1$
		
		Substituindo isso em (*) temos 
		$$(afd-bed)u_2u_1=0$$
		$$(af-be)du_2u_1=0$$
		Como S é multiplicativo e $u_1 \in S$, logo $du_2u_1 \in S$ e portanto $(a,b) \equiv (e, f)$.
	\end{enumerate}
	
	Portanto, a relação definida em $A \times S$ é de equivalência.
	
\end{proof}

\begin{notation}
	Denotamos por $S^{-1}A$ o conjunto das classes de equivalência. Denotamos por $\frac{a}{s}$ a classe de equivalência de (a,s).
\end{notation}

Vamos agora definir as operações de soma e multiplicação em $S^{-1}A$.

\begin{definition}
	A soma em $S^{-1}A$ é definida por $\frac{a}{s} + \frac{b}{t}=\frac{at+sc}{st}$
\end{definition}

\begin{definition}
	A multiplicação em $S^{-1}A$ é definida por $\frac{a}{s} . \frac{b}{t} = \frac{ab}{st}$
\end{definition}

Vamos verificar se as operações estão bem definidas, ou seja, se as operações acima não dependem do representante escolhido da classe.

\begin{proposition}
	As operações de soma e multiplicação em $S^{-1}A$ não dependem dos representantes da classe.
\end{proposition}

\begin{proof}
	Seja $\frac{a}{s}=\frac{a'}{s'}$, vamos fazer as operações com esses dois representantes e ver que a operação não depende da escolha. Sabemos que $(as'-sa')u=0$ para algum u.
	
	$\frac{a}{s} + \frac{b}{t}=\frac{at+sc}{st}$
	
	$\frac{a'}{b'} + \frac{c}{d}=\frac{a'd+b'c}{b'd}$
	
	$(ad+bc)b'd-bd(a'd+b'c)=adb'd+\cancel{bcb'd}-bda'd-\cancel{bdb'c}=adb'd-bda'd$
	
	Multiplicando tudo por u temos 
	
	$uab'dd-ua'bdd$ mas como $ab'u=ba'u$, logo $uab'dd-ua'bdd=(ab'dd-a'bdd)u=0$, ou seja, são "iguais" os resultados da soma.
	
	$\frac{a}{b} . \frac{c}{d} = \frac{ac}{bd}$
	
	$\frac{a'}{b'} . \frac{c}{d} = \frac{a'c}{b'd}$
	
	Utilizando novamente a relação temos que $acb'd-bda'c=ab'cd-a'bcd$ multiplicando por u vamos ficar com $ab'cdu-a'bcdu$ e usando $ab'u=ba'u$ ficaremos com $ab'cdu-a'bcdu=0$
\end{proof}

$S^{-1}A$ com a soma e a multiplicação definida acima é um anel.

Agora vamos ver alguns exemplos de anéis que podemos fazer essa construção. Mas antes vamos o que é um domínio de integridade, que é um tipo especial de anel.

\begin{definition}
	Um domínio de integridade (ou simplesmente domínio) é um anel comutativo unitário A tal que se $a, b \in A$ e $a\cdot b=0$ então $a=0$ ou $b=0$.
\end{definition}

\begin{example}
	O caso quando A é um domínio de integridade é um caso particular do anel de frações, isso acontece pois $S=A-\{0\}$ é um conjunto multiplicativo. 
\end{example}

Para provar o exemplo acima, basta provar a seguinte proposição.

\begin{proposition}
	Seja A um domínio de integridade, então o conjunto $C = A - \{0\}$ é um conjunto multiplicativo.
\end{proposition}

\begin{proof}
	Como $1 \in A$, logo $1 \in C$. Temos também que para $xy$ com $x,y \in C$ $xy \neq 0$, pois A é um domínio de integridade e x e y não podem ser nulos. Portanto C é fechado na multiplicação, logo é um conjunto multiplicativo.
\end{proof}

\begin{example}
	Tomando como nosso anel os inteiros ($\mathbb{Z}$) e o nosso conjunto multiplicativo como $S=\mathbb{Z} - {0}$, teremos $\frac{a}{b}$, onde  $a \in \mathbb{Z}$ e $b \in S$, ou seja, b deve ser inteiro não nulo e isso é exatamente a definição dos números racionais, que sabemos que é um corpo.
\end{example}

\begin{example}
	Podemos tomar como $A=\mathbb{Z}$ e S sendo as potências de 2, ou seja, $S=\{2^n\}$ com $n \geq 0$, dessa forma $S^{-1}A=\{\frac{a}{b}$ tal que $ a \in A$ e $b=2^n \}$ com $n \geq 0$.
\end{example}

\begin{example}
	Temos que $S^{-1}A$ será o anel zero se tivemos que $0 \in S$. De fato, pois se $0 \in S$ podemos tomar u da relação de equivalência como 0, dessa forma $(ad-bc)0=0$ para todo $\frac{a}{b}$ e $\frac{c}{d}$, dessa forma todos os elementos são equivalentes entre si, em particular serão equivalente ao elemento $\frac{0}{0}$, ou seja, $S^{-1}A$ pode ser representado por um único elemento, o $\frac{0}{0}$
\end{example}

Um importante homomorfismo é $f(a)=a/1$ para $f:A->s^{-1}A$.

f definada para um anel sem unidade

quando f é injetora? <=> sem divisor de zero

\begin{proposition} Seja $g:A -> B$ um homomorfismo de anéis tal que $g(s)$ é invertível em $B$ para todo $s \in S$. Então existe um único homomorfismo de anel $h:S^{-1}A ->B$ tal que $g = hof$
\end{proposition}




\newpage

\begin{thebibliography}{20}
	
	\bibitem{Atiyah}  Atiyah M. F.; MacDonald M. G., Introduction to Commutative Algebra . Addison-wesley publishing company, 1969.
	
	\bibitem{Fraleigh}  Fraleigh, J. B., A first course in abstract algebra. Person , 2003.
	
	\bibitem{Herstein} Herstein I. N., Topics in algebra. University of Chicago, 1975.	
	
\end{thebibliography}

\end{document}

