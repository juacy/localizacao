\documentclass[10pt,a4paper]{article}
\usepackage[utf8]{inputenc}
\usepackage[T1]{fontenc}
\usepackage{amsmath}
\usepackage{amssymb}
\usepackage{amsthm}
\usepackage{graphicx}
\usepackage{cancel}

\newcommand\ds{\displaystyle}

\newtheorem{theorem}{Teorema}[section]
\newtheorem{lemma}[theorem]{Teorema}
\newtheorem{corollary}[theorem]{Corolário}
\newtheorem{proposition}[theorem]{Proposição}
\newtheorem{example}[theorem]{Exemplo}
\newtheorem{remark}[theorem]{Observação}
\newtheorem{notation}[theorem]{Notação}
\newtheorem{definition}[theorem]{Definição}
\newtheorem{nota}[theorem]{Remark}

\begin{document}
\title {Estudos TCC}
\maketitle
\vspace{0.7cm}

\section{Anéis}
 
\definition[Anel]{Um anel R é um conjunto com duas operações $+ \textrm{ e } *$ tal que dados $x, y, z \in R$ temos:
\begin{enumerate}
	\item $x+(y+z)=(x+y)+z$
	\item $x+y \in R$
	\item $\exists 0$ tal que $\forall x, x+0=x$
	\item $x+y=y+x$
	\item $\exists -x$ tal que $ x, x+(-x)=x$
	\item $a*b \in R$
	\item $(a*b)*c=a*(b*c)$
	\item $a*(b+c)=a*b+a*c$
	\item $\exists 1 \textrm{ tal que } a*1=1*a=a$
\end{enumerate}
}

\remark{A definição de anel varia de autor para autor, alguns consideram anéis comutativos com unidade, outro já não consideram a existência do neutro multiplicativo. Também existem anéis sem associatividade, então podemos ter um anel sem nenhuma propriedade sobre a multiplicação}

Quando a multiplicação também é comutativa, isto é $a*b=b*a$ chamamos de anel comutativo.

\example{Temos que $\mathbb{Z},\mathbb{Q}, \mathbb{R} $ com as operações usuais de soma e produto são anéis.}

\proposition{Seja R um anel não trivial, se $x \in R$ então $x*0=0$}
\begin{proof}
Temos que $$x*0=x*(0+0)=x*0+x*0$$ mas como existe o oposto de $x*0$ podemos somar
$$x*0-x*0=x*0+x*0-x*0$$ logo $$0=x*0$$
\end{proof}

Temos que caso $1=0$ o anel é trivial. Já que $x=x*1=x*0=0$.

\definition[Subaneis] Seja $S \subseteq R$, onde $R$ é um anel, dizemos que $S$ é um subanel de $R$ se dados $x,y \in S$ temos:
\begin{enumerate}
	\item $x+y \in S$
	\item Se $x \in S$ então $-x \in S$
	\item $0 \in S$
	\item $xy \in S$
\end{enumerate}

\definition[Homomorfismo] Uma função $f:A \rightarrow B $ é dita homomorfismo de anéis se:
\begin{enumerate}
	\item $f(x+y)=f(x)+f(y)$
	\item $f(xy)=f(x)f(y)$
\end{enumerate}
 
\definition[Ideais(maximal e primo)] Colocar a definição de ideal, ideal maximal, ideal primo

\definition[Divisores de zero] Colocar a definição de divisores de zero

\definition[Anéis quociente] Fazer a construção de quociente de anéis


\section{Corpo de fração}

\section{Localização em anéis comutativos}

Vamos seguir a demonstração a partir de um anel $A$ sem unidade, na referência \cite{Atiyah} temos a demonstração feita em anel com unidade.

\begin{definition}
	Um subconjunto S de um anel A é dito um conjunto multiplicativo se $1 \in S$ e $ x\cdot y \in S$ para todo $x , y \in S$.
\end{definition}

Mas como estamos partindo de um anel sem unidade, não faz sentido querer que $1 \in S$, então para o nosso caso vamos considerar que exista $a_s \neq 0$ tal que $a \in S$.

Agora que já temos todas as definições necessárias para iniciar a construção, vamos começar.

\begin{definition}
	Seja A um anel e S um conjunto multiplicativo de A. Vamos definir uma relação em $A \times S$ como $$(a,s) \equiv (b, t) \Leftrightarrow (at-sb)u=0$$ para algum $u \in S$.
\end{definition}

Esta relação, é uma relação de equivalência.

\begin{proof} Vamos mostrar que é uma relação reflexiva, simétrica e transitiva.
	
	\begin{enumerate}
		\item [Reflexiva]  $(a,s) \equiv (a, s)$, de fato, pois $as=sa$, já que estamos trabalhando com um anel comutativo, portanto $as-sa=0$ e assim $(as-sa)a_s=0$.
		\item [Simétrica] Temos que $(a,s) \equiv (b, t)$ nos leva a $(at-sb)u=0$. Note que podemos somar o inverso aditivo do elemento $(at-sb)u$ em ambos lados da igualdade que nos leva a 
		$$(at-sb)u-((at-sb)u)=-(at-sb)u$$
		$$0=-(at-sb)u$$
		$$0=(-at+sb)u$$
		Como o anel A é comutativo
		$$0=(sb-at)u$$
		Usando a comutatividade novamente para reorganizar os produtos
		$$0=(bs-ta)u$$
		que é equivalente a 
		$$(b,t) \equiv (a, s)$$
		\item [Transitividade] Seja $(a,s) \equiv (b, t)$ e $(b,t) \equiv (c, r)$, devemos chegar em $(a,s) \equiv (c, r)$.
		\\
		
		De $(a,s) \equiv (b, t)$ temos $(at-sb)u_1=0$ para algum $u_1 \in S$.
		
		De $(b,t) \equiv (c, r)$ temos $(br-tc)u_2=0$ para algum $u_2 \in S$.
		
		Multiplicando a primeira equação por $ru_2$ e a segunda por $su_1$ chegamos a 
		
		$$ru_2(at-sb)u_1=0$$
		$$su_1(br-tc)u_2=0$$
		
		Utilizando a comutatividade para agrupar os termos em u.
		
		$$r(at-sb)u_1u_2=0$$
		$$s(br-tc)u_1u_2=0$$
		
		Aplicando a propriedade distributiva
		
		$$(rat-rsb)u_1u_2=0$$
		$$(sbr-stc)u_1u_2=0$$
		
		Novamente utilizando a comutatividade para ajustar os termos
		
		$$(art-sbr)u_1u_2=0 (*)$$
		$$(sbr-sct)u_1u_2=0 (**)$$
		
		Em (**) temos que ao aplicar a distributiva
		
		$$sbru_1u_2=sctu_1u_2$$
		
		Mas note que temos em (*) ao aplicar a distributiva 
		
		$$artu_1u_2=sbru_1u_2 (*)$$
		
		Então temos que 
		
		$$artu_1u_2=sctu_1u_2$$
		$$artu_1u_2-sctu_1u_2=$$
		$$(art-sct)u_1u_2$$
		$$(ar-sc)tu_1u_2$$
		
		Mas como $t, u_1, u_2 \in S$ e $S$ é um conjunto fechado para a multiplicação, logo $tu_1u_2 \in S$ e portanto $(a,s) \equiv (c, r)$.
		
	\end{enumerate}
	
	Portanto, a relação definida em $A \times S$ é de equivalência.
	
\end{proof}

\begin{notation}
	Denotamos por $S^{-1}A$ o conjunto das classes de equivalência. Denotamos por $\frac{a}{s}$ a classe de equivalência de $(a,s)$.
\end{notation}

Vamos agora definir as operações de soma e multiplicação em $S^{-1}A$.

\begin{definition}
	A soma em $S^{-1}A$ é definida por $(a,s)+(b,t)=(at+sb,st)$
\end{definition}

\begin{definition}
	A multiplicação em $S^{-1}A$ é definida por $(a,s)*(b,t)=(ab,st)$
\end{definition}

Vamos verificar se as operações estão bem definidas, ou seja, se as operações acima não dependem do representante da classe.

\begin{proposition}
	As operações de soma e multiplicação em $S^{-1}A$ não dependem dos representantes da classe.
\end{proposition}

\begin{proof}
	Seja $(a,s)=(a',s')$, vamos fazer as operações com esses dois representantes e ver que a operação não depende da escolha. Sabemos que $(as'-sa')u=0$ para algum $u$.
	
	\begin{enumerate}
		\item[Soma] Vamos tomar as somas de  $(a,s)$ e $(a',s')$ com $(b,t)$.
		
		$$(a,s)+(b,t)=(at+sb,st)$$
		$$(a',s')+(b,t)=(a't+s'b,s't)$$
		
		Quero mostrar que $(a,s)+(b,t)=(a',s')+(b,t)$, para isso vamos fazer 
		
		$$(at+sb,st)=(a't+s'b,s't)$$
		
		$$(at+sb)s't-st(a't+s'b)$$
		
		$$ats't+sbs't-sta't-sts'b$$
		
		Usando a comutatividade 
		
		$$atts'+bsts'-stta'-bsts'=atts'-stta'=(as'-sa')tt$$
		
		Multiplicando por $u$
		
		$$(as'-sa')ttu$$
		
		Mas como 
		
		$$(as'-sa')u=0$$
		
		Logo
		
		$$(as'-sa')ttu=0$$
		
		Portanto 
		
		$$(a,s)+(b,t)=(a',s')+(b,t)$$
		
		\item[Produto] Vamos tomar os produtos de  $(a,s)$ e $(a',s')$ com $(b,t)$.
		
		$$(a,s)*(b,t)=(ab,st)$$
		$$(a',s')*(b,t)=(a'b,s't)$$
		
		Quero mostrar que 
		
		$$(ab,st)=(a'b,s't)$$
		
		Então tome 
		
		$$abs't-sta'b$$
		
		Utilizando a comutatividade para reorganizar
		
		$$as'bt-sa'bt$$
		
		Colocando em evidência
		
		$$(as'-sa')bt$$
		
		Multiplicando por $u$ temos 
		
		$$(as'-sa')ubt$$
		
		Mas sabemos que 
		
		$$(as'-sa')u=0$$
		
		Portanto 
		
		$$(as'-sa')ubt=0$$
		
		Logo $$(ab,st)=(a'b,s't)$$
		
	\end{enumerate}
	
	
	
	
\end{proof}

$S^{-1}A$ com a soma e a multiplicação definidas acima é um anel.

Agora vamos ver alguns exemplos de anéis que podemos fazer essa construção. Mas antes vamos o que é um domínio de integridade, que é um tipo especial de anel.

\begin{definition}
	Um domínio de integridade (ou simplesmente domínio) é um anel comutativo unitário A tal que se $a, b \in A$ e $a\cdot b=0$ então $a=0$ ou $b=0$.
\end{definition}

\begin{example}
	O caso quando A é um domínio de integridade é um caso particular do anel de frações, isso acontece pois $S=A-\{0\}$ é um conjunto multiplicativo. 
\end{example}

Para provar o exemplo acima, basta provar a seguinte proposição.

\begin{proposition}
	Seja A um domínio de integridade, então o conjunto $C = A - \{0\}$ é um conjunto multiplicativo.
\end{proposition}

\begin{proof}
	Como $1 \in A$, logo $1 \in C$. Temos também que para $xy$ com $x,y \in C$ $xy \neq 0$, pois A é um domínio de integridade e x e y não podem ser nulos. Portanto C é fechado na multiplicação, logo é um conjunto multiplicativo.
\end{proof}

\begin{example}
	Tomando como nosso anel os inteiros ($\mathbb{Z}$) e o nosso conjunto multiplicativo como $S=\mathbb{Z} - {0}$, teremos $\frac{a}{b}$, onde  $a \in \mathbb{Z}$ e $b \in S$, ou seja, b deve ser inteiro não nulo e isso é exatamente a definição dos números racionais, que sabemos que é um corpo.
\end{example}

\begin{example}
	Podemos tomar como $A=\mathbb{Z}$ e S sendo as potências de 2, ou seja, $S=\{2^n\}$ com $n \geq 0$, dessa forma $S^{-1}A=\{\frac{a}{b}$ tal que $ a \in A$ e $b=2^n \}$ com $n \geq 0$.
\end{example}

\begin{example}
	Temos que $S^{-1}A$ será o anel zero se tivemos que $0 \in S$. De fato, pois se $0 \in S$ podemos tomar u da relação de equivalência como 0, dessa forma $(ad-bc)0=0$ para todo $\frac{a}{b}$ e $\frac{c}{d}$, dessa forma todos os elementos são equivalentes entre si, em particular serão equivalente ao elemento $\frac{0}{0}$, ou seja, $S^{-1}A$ pode ser representado por um único elemento, o $\frac{0}{0}$
\end{example}

Um importante homomorfismo é $f(a)=(a,1)$ para $f:A \rightarrow s^{-1}A$. Mas vamos definir esse homomorfismo para um anel $A$ sem unidade, então queremos algo como $f(a)=(as,s)$ onde $s \in S$ com $s \neq 0$.

\begin{definition}
	Seja $A$ um anel associativo, comutativo mas não necessariamente com unidade e $S$ um conjunto multiplicativo não vazio, então definimos o seguinte homomorfismo $f:A \rightarrow S^{-1}A$ como $f(a)=(as,s)$ onde $s \in S$ com $s \neq 0$.
\end{definition}

\begin{proposition}
	$f:A \rightarrow S^{-1}A$ como $f(a)=(as,s)$ onde $s \in S$ com $s \neq 0$ é um homomorfismo.
\end{proposition}
\begin{proof}
	\begin{enumerate}
		\item[Soma]Queremos mostrar que$f(a+b)=f(a)+f(b)$
		$$f(a+b)=((a+b)s,s)=(as+bs,s)$$
		\item[Produto]
	\end{enumerate}
\end{proof}

quando f é injetora? <=> sem divisor de zero

\begin{proposition} Seja $g:A \rightarrow B$ um homomorfismo de anéis tal que $g(s)$ é invertível em $B$ para todo $s \in S$. Então existe um único homomorfismo de anel $h:S^{-1}A \rightarrow B$ tal que $g = h\circ f$
\end{proposition}

\begin{proof}
	 Vamos definir $h(a,s)=g(a)g(s)^{-1}$
	 
	 Primeiro, vamos verificar que $h$ está bem definida, ou seja, não depende do representante da classe escolhido.
	 
	 $$(a,s)=(b,t)$$
	 
	 Que significa $$(at-sb)u=0 \text{ para algum } u \in S$$
	 
	 Mas como sabemos $g$ é um homomorfismo de $A$ para $B$ e todos elementos estão em $A$, logo podemos aplicar $g$ na igualdade.
	 
	 $$g((at-sb)u)=g(0)$$
	 $$g(at-sb)g(u)=0$$
	 $$g(at-sb)g(u)g(u)^{-1}=0g(u)^{-1}$$
	 $$g(at-sb)=0$$
	 $$g(a)g(t)-g(s)g(b)=0$$	 
	 $$g(a)g(t)=g(s)g(b)$$
	 $$g(s)^{-1}g(a)g(t)g(t)^{-1}=g(s)^{-1}g(s)g(b)g(t)^{-1}$$
	 $$g(s)^{-1}g(a)=g(b)g(t)^{-1}$$
	 
	 Como estamos trabalhando com anéis comutativos
	 
	 $$h(a,s)=g(a)g(s)^{-1}=g(b)g(t)^{-1}=h(b,t)$$
	 
	 Ou seja, $h$ não depende dos representantes escolhidos.
	 
	 Agora vamos mostrar que $h$ é um homomorfismo.
	 
	 \begin{enumerate}
	 	\item[soma] Queremos mostrar que $h((a,s)+(b,t))=h(a,s)+h(b,t)$ 
	 	Sabemos que
	 	$$h((a,s)+(b,t))=h(at+sb,st)=g(at+sb)g(st)^{-1}$$
	 	$$g(at+sb)g(st)^{-1}=[g(a)g(t)+g(s)g(b)][g(s)g(t)]^{-1}$$
	 	$$[g(a)g(t)+g(s)g(b)]g(s)^{-1}g(t)^{-1}$$
	 	$$g(a)g(t)g(s)^{-1}g(t)^{-1}+g(s)g(b)g(s)^{-1}g(t)^{-1}$$
	 	Utilizando a comutatividade do anel $B$
	 	$$g(a)g(t)g(t)^{-1}g(s)^{-1}+g(b)g(s)g(s)^{-1}g(t)^{-1}$$
	 	$$g(a)g(s)^{-1}+g(b)g(t)^{-1}=h(a,s)+h(b,t)$$
	 	\item[produto] Queremos mostrar que $h((a,s)(b,t))=h(a,s)h(b,t)$
	 	Sabemos que
	 	$$h((a,s)(b,t))=h(ab,st)$$
	 	$$h(ab,st)=g(ab)g(st)^{-1}$$
	 	$$g(ab)g(st)^{-1}=g(ab)[g(st)]^{-1}$$
	 	$$g(a)g(b)g(s)^{-1}g(t)^{-1}$$
	 	Como o anel $B$ é comutativo, temos que
	 	$$g(a)g(s)^{-1}g(b)g(t)^{-1}=h(a,s)h(b,t)$$	 	
	 \end{enumerate}
 
 	Portanto, temos que $h$ é homomorfismo.
 
 	Agora, basta mostrar que $g = h\circ f$.
 	
 	Temos que $$ h\circ f=h(f(a))=h(as,s)$$
 	$$h(as,s)=g(as)g(s)^{-1}=g(a)g(s)g(s)^{-1}=g(a)$$
 	
 	Portanto a composição se verifica.
	 
\end{proof}




\newpage

\begin{thebibliography}{20}
	
	\bibitem{Atiyah}  Atiyah M. F.; MacDonald M. G., Introduction to Commutative Algebra . Addison-wesley publishing company, 1969.
	
	\bibitem{Fraleigh}  Fraleigh, J. B., A first course in abstract algebra. Person , 2003.
	
	\bibitem{Herstein} Herstein I. N., Topics in algebra. University of Chicago, 1975.	
	
\end{thebibliography}

\end{document}

