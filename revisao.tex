\documentclass[10pt,a4paper]{article}
\usepackage[utf8]{inputenc}
\usepackage[T1]{fontenc}
\usepackage{amsmath}
\usepackage{amssymb}
\usepackage{amsthm}
\usepackage{graphicx}
\usepackage{cancel}

\newcommand\ds{\displaystyle}

\newtheorem{theorem}{Teorema}[section]
\newtheorem{lemma}[theorem]{Teorema}
\newtheorem{corollary}[theorem]{Corolário}
\newtheorem{proposition}[theorem]{Proposição}
\newtheorem{example}[theorem]{Exemplo}
\newtheorem{remark}[theorem]{Observação}
\newtheorem{notation}[theorem]{Notação}
\newtheorem{definition}[theorem]{Definição}
\newtheorem{nota}[theorem]{Remark}

\begin{document}
\title {Estudos TCC}
\maketitle
\vspace{0.7cm}

\section{Anéis}
 
\definition[Anel]{Um anel R é um conjunto com duas operações $+ \textrm{ e } *$ tal que dados $x, y, z \in R$ temos:
\begin{enumerate}
	\item $x+(y+z)=(x+y)+z$
	\item $x+y \in R$
	\item $\exists 0$ tal que $\forall x, x+0=x$
	\item $x+y=y+x$
	\item $\exists -x$ tal que $ x, x+(-x)=x$
	\item $a*b \in R$
	\item $(a*b)*c=a*(b*c)$
	\item $a*(b+c)=a*b+a*c$
	\item $\exists 1 \textrm{ tal que } a*1=1*a=a$
\end{enumerate}
}

\remark{A definição de anel varia de autor para autor, alguns consideram anéis comutativos com unidade, outro já não consideram a existência do neutro multiplicativo. Também existem anéis sem associatividade, então podemos ter um anel sem nenhuma propriedade sobre a multiplicação}

Quando a multiplicação também é comutativa, isto é $a*b=b*a$ chamamos de anel comutativo.

\example{Temos que $\mathbb{Z},\mathbb{Q}, \mathbb{R} $ com as operações usuais de soma e produto são anéis.}

\proposition{Seja R um anel não trivial, se $x \in R$ então $x*0=0$}
\begin{proof}
Temos que $$x*0=x*(0+0)=x*0+x*0$$ mas como existe o oposto de $x*0$ podemos somar
$$x*0-x*0=x*0+x*0-x*0$$ logo $$0=x*0$$
\end{proof}

Temos que caso $1=0$ o anel é trivial. Já que $x=x*1=x*0=0$.

\definition[Subaneis] Seja $S \subseteq R$, onde $R$ é um anel, dizemos que $S$ é um subanel de $R$ se dados $x,y \in S$ temos:
\begin{enumerate}
	\item $x+y \in S$
	\item Se $x \in S$ então $-x \in S$
	\item $0 \in S$
	\item $xy \in S$
\end{enumerate}

\definition[Homomorfismo] Uma função $f:A \rightarrow B $ é dita homomorfismo de anéis se:
\begin{enumerate}
	\item $f(x+y)=f(x)+f(y)$
	\item $f(xy)=f(x)f(y)$
\end{enumerate}

\begin{definition}Seja R um anel. $I \subseteq R$ com $I \neq \emptyset$ dizemos que I é um ideal de R se:
	\begin{enumerate}
		\item $0_R \in I$
		\item $x,y \in I$ então $x+y \in I$
		\item $a \in R$ e $x \in I$ então $ax \in I$
	\end{enumerate}
\end{definition} 
 
\begin{definition}
	[Ideais(maximal e primo)] Colocar a definição de ideal, ideal maximal, ideal primo
\end{definition}

\begin{definition}
	Seja $R$ um anel e $S \subseteq R$ com $S \neq \emptyset$, dizemos que o ideal gerado pelo conjunto S é $<S>=\{ \sum_{i=1}^{n} a_i s_i | n \geq 0, a_i \in R, s_i \in S, 1 \leq i \leq n\}$.
\end{definition}

\begin{proposition}
	Seja $R$ um anel e $S \subseteq R$ então <S> é um ideal de R.
\end{proposition}
\begin{proof}
	\begin{enumerate}
		\item $0_R \in <S>$.
		
		Temos que $S \neq \emptyset$, tome $x \in S$ temos que $0_R x = 0_R \in <S>$.
		
		\item $x, y \in <S>$ então $x+y \in <S>$.
		
		Temos então que $x=\sum_{i=1}^{k}a_i x_i$ e $y=\sum_{j=1}^{m}b_j y_j$ onde $a_i, b_j \in R$ e $x_i, y_j \in S$.
		
		Temos então $x+y=\sum_{i=1}^{k}a_i x_i + \sum_{j=1}^{m}b_j y_j = \sum_{l=1}^{k+m} c_l z_l$ onde $c_l = a_i$, $z_l = x_i$  quando $1 \leq l \leq k$ e $c_l = b_i$, $z_l = y_i$  quando $k+1 \leq l \leq k+m$ portanto $x+y \in <S>$
		
		\item $a \in R$ e $x\in <S>$ então $x+y \in <S>$.
		
		Temos então que $x=\sum_{i=1}^{k}a_i x_i$, logo $ax=a\sum_{i=1}^{k}a_i x_i = \sum_{i=1}^{k}aa_i x_i \in <S>$.
	\end{enumerate}

	Portanto <S> é um ideal de R.
\end{proof}

\begin{proposition}
	Seja $f:A \rightarrow B$ um homomorfismo de anéis e J um ideal de B então $f^{-1}(J)$ é um ideal de A. Ou seja, a pré-imagem de um ideal é um ideal.
\end{proposition}
\begin{proof}
	Primeiro vamos a definição da pré-imagem de J. $$f^{-1}(J)=\{x \in A | f(x) \in J\}$$
	
	\begin{enumerate}
		\item $0_A \in f^{-1}(J)$
		
		Temos que J é um ideal de B, então é claro que $0_B \in J$. Mas como $f$ é um homomorfismo então temos que $f(0_A)=0_B$, logo $0_A \in f^{-1}(J)$.
		
		\item $x,y \in f^{-1}(J)$ então $x+y \in f^{-1}(J)$
		
		Como $x,y \in f^{-1}(J)$ então temos que $f(x),f(y) \in J$. Como J é um ideal então $f(x)+f(y) \in J$. Temos também que f é um homomorfismo de anéis então $f(x)+f(y)=f(x+y) \in J$, logo $x+y \in f^{-1}(J)$. 
		
		\item $a \in A$ e $x \in f^{-1}(J)$ então $ax \in f^{-1}(J)$
		
		Como $x \in f^{-1}(J)$ então temos que $f(x) \in J$, como J é ideal então $f(a) \in B$ e logo $f(a)f(x) \in J$ e $f(a)f(x)=f(ax)$ pois f é um homomorfismo portanto $ax \in f^{-1}(J)$.
	\end{enumerate}

	Portanto a pré imagem de um ideal é um ideal no anel do domínio do homomorfismo.
\end{proof}

	Seja $f:A \rightarrow B$ um homomorfismo de anéis e I um ideal de A, não podemos garantir que $f(I)$ é um ideal de B.
	
	Tome $i:\mathbb{Z} \rightarrow \mathbb{Q}$ sendo $i(x)=x$. Temos que $\mathbb{Q}$ é um corpo, portanto seus únicos ideias são $\{0\}$ e $\mathbb{Q}$. Tome $2\mathbb{Z}$ ideal de $\mathbb{Z}$ temos que $i(2\mathbb{Z}) \neq \{0\}$ e $i(2\mathbb{Z}) \neq \mathbb{Q}$ portanto não é um ideal de $\mathbb{Q}$.
	
\begin{definition}
	Seja $f:A \rightarrow B$ um homomorfismo de anéis e I um ideal de A. Dizemos que a extensão de I, denotada por $I^{e}$ é o ideal gerado por $<f(I)>$. Ou seja, é um ideal em B.
\end{definition}

\begin{definition}
	Seja $f:A \rightarrow B$ um homomorfismo de anéis e J um ideal de B. Dizemos que a contração de J, denotada por $J^{c}$ é $<f^{-1}(J)>$. Ou seja, é um ideal em A.
\end{definition}


\definition[Divisores de zero] Colocar a definição de divisores de zero

\definition[Anéis quociente] Fazer a construção de quociente de anéis


\subsection{Corpo de fração}

\subsection{Localização em anéis comutativos}

Vamos seguir a demonstração a partir de um anel $A$ sem unidade, na referência \cite{Atiyah} temos a demonstração feita em anel com unidade.

\begin{definition}
	Um subconjunto S de um anel A é dito um conjunto multiplicativo se $1 \in S$ e $ x\cdot y \in S$ para todo $x , y \in S$.
\end{definition}

Mas como estamos partindo de um anel sem unidade, não faz sentido querer que $1 \in S$, então para o nosso caso vamos considerar que exista $a_s \neq 0$ tal que $a \in S$.

Agora que já temos todas as definições necessárias para iniciar a construção, vamos começar.

\begin{definition}
	Seja A um anel e S um conjunto multiplicativo de A. Vamos definir uma relação em $A \times S$ como $$(a,s) \equiv (b, t) \Leftrightarrow (at-sb)u=0$$ para algum $u \in S$.
\end{definition}

Esta relação, é uma relação de equivalência.

\begin{proof} Vamos mostrar que é uma relação reflexiva, simétrica e transitiva.
	
	\begin{enumerate}
		\item [Reflexiva]  $(a,s) \equiv (a, s)$, de fato, pois $as=sa$, já que estamos trabalhando com um anel comutativo, portanto $as-sa=0$ e assim $(as-sa)a_s=0$.
		\item [Simétrica] Temos que $(a,s) \equiv (b, t)$ nos leva a $(at-sb)u=0$. Note que podemos somar o inverso aditivo do elemento $(at-sb)u$ em ambos lados da igualdade que nos leva a 
		$$(at-sb)u-((at-sb)u)=-(at-sb)u$$
		$$0=-(at-sb)u$$
		$$0=(-at+sb)u$$
		Como o anel A é comutativo
		$$0=(sb-at)u$$
		Usando a comutatividade novamente para reorganizar os produtos
		$$0=(bs-ta)u$$
		que é equivalente a 
		$$(b,t) \equiv (a, s)$$
		\item [Transitividade] Seja $(a,s) \equiv (b, t)$ e $(b,t) \equiv (c, r)$, devemos chegar em $(a,s) \equiv (c, r)$.
		\\
		
		De $(a,s) \equiv (b, t)$ temos $(at-sb)u_1=0$ para algum $u_1 \in S$.
		
		De $(b,t) \equiv (c, r)$ temos $(br-tc)u_2=0$ para algum $u_2 \in S$.
		
		Multiplicando a primeira equação por $ru_2$ e a segunda por $su_1$ chegamos a 
		
		$$ru_2(at-sb)u_1=0$$
		$$su_1(br-tc)u_2=0$$
		
		Utilizando a comutatividade para agrupar os termos em u.
		
		$$r(at-sb)u_1u_2=0$$
		$$s(br-tc)u_1u_2=0$$
		
		Aplicando a propriedade distributiva
		
		$$(rat-rsb)u_1u_2=0$$
		$$(sbr-stc)u_1u_2=0$$
		
		Novamente utilizando a comutatividade para ajustar os termos
		
		$$(art-sbr)u_1u_2=0 (*)$$
		$$(sbr-sct)u_1u_2=0 (**)$$
		
		Em (**) temos que ao aplicar a distributiva
		
		$$sbru_1u_2=sctu_1u_2$$
		
		Mas note que temos em (*) ao aplicar a distributiva 
		
		$$artu_1u_2=sbru_1u_2 (*)$$
		
		Então temos que 
		
		$$artu_1u_2=sctu_1u_2$$
		$$artu_1u_2-sctu_1u_2=$$
		$$(art-sct)u_1u_2$$
		$$(ar-sc)tu_1u_2$$
		
		Mas como $t, u_1, u_2 \in S$ e $S$ é um conjunto fechado para a multiplicação, logo $tu_1u_2 \in S$ e portanto $(a,s) \equiv (c, r)$.
		
	\end{enumerate}
	
	Portanto, a relação definida em $A \times S$ é de equivalência.
	
\end{proof}

\begin{notation}
	Denotamos por $S^{-1}A$ o conjunto das classes de equivalência. Denotamos por $\frac{a}{s}$ a classe de equivalência de $(a,s)$.
\end{notation}

Vamos agora definir as operações de soma e multiplicação em $S^{-1}A$.

\begin{definition}
	A soma em $S^{-1}A$ é definida por $(a,s)+(b,t)=(at+sb,st)$
\end{definition}

\begin{definition}
	A multiplicação em $S^{-1}A$ é definida por $(a,s)*(b,t)=(ab,st)$
\end{definition}

Vamos verificar se as operações estão bem definidas, ou seja, se as operações acima não dependem do representante da classe.

\begin{proposition}
	As operações de soma e multiplicação em $S^{-1}A$ não dependem dos representantes da classe.
\end{proposition}

\begin{proof}
	Seja $(a,s)=(a',s')$, vamos fazer as operações com esses dois representantes e ver que a operação não depende da escolha. Sabemos que $(as'-sa')u=0$ para algum $u$.
	
	\begin{enumerate}
		\item[Soma] Vamos tomar as somas de  $(a,s)$ e $(a',s')$ com $(b,t)$.
		
		$$(a,s)+(b,t)=(at+sb,st)$$
		$$(a',s')+(b,t)=(a't+s'b,s't)$$
		
		Quero mostrar que $(a,s)+(b,t)=(a',s')+(b,t)$, para isso vamos fazer 
		
		$$(at+sb,st)=(a't+s'b,s't)$$
		
		$$(at+sb)s't-st(a't+s'b)$$
		
		$$ats't+sbs't-sta't-sts'b$$
		
		Usando a comutatividade 
		
		$$atts'+bsts'-stta'-bsts'=atts'-stta'=(as'-sa')tt$$
		
		Multiplicando por $u$
		
		$$(as'-sa')ttu$$
		
		Mas como 
		
		$$(as'-sa')u=0$$
		
		Logo
		
		$$(as'-sa')ttu=0$$
		
		Portanto 
		
		$$(a,s)+(b,t)=(a',s')+(b,t)$$
		
		\item[Produto] Vamos tomar os produtos de  $(a,s)$ e $(a',s')$ com $(b,t)$.
		
		$$(a,s)*(b,t)=(ab,st)$$
		$$(a',s')*(b,t)=(a'b,s't)$$
		
		Quero mostrar que 
		
		$$(ab,st)=(a'b,s't)$$
		
		Então tome 
		
		$$abs't-sta'b$$
		
		Utilizando a comutatividade para reorganizar
		
		$$as'bt-sa'bt$$
		
		Colocando em evidência
		
		$$(as'-sa')bt$$
		
		Multiplicando por $u$ temos 
		
		$$(as'-sa')ubt$$
		
		Mas sabemos que 
		
		$$(as'-sa')u=0$$
		
		Portanto 
		
		$$(as'-sa')ubt=0$$
		
		Logo $$(ab,st)=(a'b,s't)$$
		
	\end{enumerate}
	
	
	
	
\end{proof}

$S^{-1}A$ com a soma e a multiplicação definidas acima é um anel.

Agora vamos ver alguns exemplos de anéis que podemos fazer essa construção. Mas antes vamos o que é um domínio de integridade, que é um tipo especial de anel.

\begin{definition}
	Um domínio de integridade (ou simplesmente domínio) é um anel comutativo unitário A tal que se $a, b \in A$ e $a\cdot b=0$ então $a=0$ ou $b=0$.
\end{definition}

\begin{example}
	O caso quando A é um domínio de integridade é um caso particular do anel de frações, isso acontece pois $S=A-\{0\}$ é um conjunto multiplicativo. 
\end{example}

Para provar o exemplo acima, basta provar a seguinte proposição.

\begin{proposition}
	Seja A um domínio de integridade, então o conjunto $C = A - \{0\}$ é um conjunto multiplicativo.
\end{proposition}

\begin{proof}
	Como $1 \in A$, logo $1 \in C$. Temos também que para $xy$ com $x,y \in C$ $xy \neq 0$, pois A é um domínio de integridade e x e y não podem ser nulos. Portanto C é fechado na multiplicação, logo é um conjunto multiplicativo.
\end{proof}

\begin{example}
	Tomando como nosso anel os inteiros ($\mathbb{Z}$) e o nosso conjunto multiplicativo como $S=\mathbb{Z} - {0}$, teremos $\frac{a}{b}$, onde  $a \in \mathbb{Z}$ e $b \in S$, ou seja, b deve ser inteiro não nulo e isso é exatamente a definição dos números racionais, que sabemos que é um corpo.
\end{example}

\begin{example}
	Podemos tomar como $A=\mathbb{Z}$ e S sendo as potências de 2, ou seja, $S=\{2^n\}$ com $n \geq 0$, dessa forma $S^{-1}A=\{\frac{a}{b}$ tal que $ a \in A$ e $b=2^n \}$ com $n \geq 0$.
\end{example}

\begin{example}
	Temos que $S^{-1}A$ será o anel zero se tivemos que $0 \in S$. De fato, pois se $0 \in S$ podemos tomar u da relação de equivalência como 0, dessa forma $(ad-bc)0=0$ para todo $\frac{a}{b}$ e $\frac{c}{d}$, dessa forma todos os elementos são equivalentes entre si, em particular serão equivalente ao elemento $\frac{0}{0}$, ou seja, $S^{-1}A$ pode ser representado por um único elemento, o $\frac{0}{0}$
\end{example}

\begin{remark}
	Um importante homomorfismo é $f(a)=(a,1)$ para $f:A \rightarrow s^{-1}A$. Mas vamos definir esse homomorfismo para um anel $A$ sem unidade, então queremos algo como $f(a)=(as,s)$ onde $s \in S$ com $s \neq 0$. Na referência \cite{Atiyah} temos o resultado para um anel com unidade.
\end{remark}

\begin{definition}
	Seja $A$ um anel associativo, comutativo mas não necessariamente com unidade e $S$ um conjunto multiplicativo não vazio, então definimos o seguinte homomorfismo $f:A \rightarrow S^{-1}A$ como $f(a)=(as,s)$ onde $s \in S$ com $s \neq 0$.
\end{definition}

\begin{proposition}
	$f:A \rightarrow S^{-1}A$ como $f(a)=(as,s)$ onde $s \in S$ com $s \neq 0$ é um homomorfismo.
\end{proposition}
\begin{proof}
	\begin{enumerate}
		\item[Soma]Queremos mostrar que$f(a+b)=f(a)+f(b)$
		$$f(a+b)=((a+b)s,s)=(as+bs,s)$$
		$$f(a)+f(b)=(as,s)+(bs,s)=(ass+bss,ss)$$
		Como estamos trabalhando com classes de equivalência, devo mostrar que as classes $(as+bs,s)$ e $(ass+bss,ss)$ são equivalentes. Note que 
		$$(as+bs,s)=(ass+bss,ss)$$
		Se e somente se
		$$[(as+bs)ss-(ass+bss)s]u \textrm{ para algum } u \in S$$
		Mas note que 
		$$[(as+bs)ss-(ass+bss)s]=asss+bsss-asss-bsss=0$$
		Portanto, podemos escolher qualquer $u \in S$, em particular vamos tomar $s$
		$$[(as+bs)ss-(ass+bss)s]s=0$$
		Portanto
		$$(as+bs,s)=(ass+bss,ss)$$
		\item[Produto]Queremos mostrar que$f(ab)=f(a)f(b)$
		$$f(ab)=((ab)s,s)=(abs,s)$$
		$$f(a)f(b)=(as,s)(bs,s)=(asbs,ss)$$
		Aplicando uma estratégia semelhante ao que fizemos para a soma, queremos mostrar que $(abs,s)=(asbs,ss)$.
		Para isto ocorrer, devemos ter 
		$$(absss-asbss)u \textrm{ para algum } u \in S$$
		Mas note que pela comutatividade temos que
		$$absss-asbss=absss-absss=0$$
		Logo podemos tomar $u=s$ e assim $(abs,s)=(asbs,ss)$.
	\end{enumerate}

	Portanto $f$ é um homomorfismo.
\end{proof}

\begin{proposition}
	Seja $f:A \rightarrow S^{-1}A$ como $f(a)=(as,s)$ onde $s \in S$ com $s \neq 0$ um homomorfismo de anéis. $f$ é um homomorfismo injetor se e somente se $S$ não possui divisores de zero e $0 \notin S$.
\end{proposition}

\begin{proof}
	$\Leftarrow$ Temos que $f$ é injetora se $f(a)=f(b)$ então $a=b$.
	$$f(a)=f(b)=(as,s)=(bs,s)$$
	Logo temos que
	$$(ass-sbs)u=0 \textrm{ para algum } u \in S$$
	Utilizando a comutatividade de $A$ e colocando $s$ em evidência temos que
	$$(a-b)ssu=0 \textrm{ para algum } u \in S$$
	Note que como $S$ não possui divisores de zero e nem o elemento nulo, logo $ssu \neq 0$ e assim $(a-b)=0$ logo $(a=b)$, portanto $f$ é injetora.	
	\newline
	$\Rightarrow$ Fazer a volta
\end{proof}

\begin{proposition} Seja $g:A \rightarrow B$ um homomorfismo de anéis tal que $g(s)$ é invertível em $B$ para todo $s \in S$. Então existe um único homomorfismo de anel $h:S^{-1}A \rightarrow B$ tal que $g = h\circ f$
\end{proposition}

\begin{proof}
	 Vamos definir $h(a,s)=g(a)g(s)^{-1}$
	 
	 Primeiro, vamos verificar que $h$ está bem definida, ou seja, não depende do representante da classe escolhido.
	 
	 $$(a,s)=(b,t)$$
	 
	 Que significa $$(at-sb)u=0 \text{ para algum } u \in S$$
	 
	 Mas como sabemos $g$ é um homomorfismo de $A$ para $B$ e todos elementos estão em $A$, logo podemos aplicar $g$ na igualdade.
	 
	 $$g((at-sb)u)=g(0)$$
	 $$g(at-sb)g(u)=0$$
	 $$g(at-sb)g(u)g(u)^{-1}=0g(u)^{-1}$$
	 $$g(at-sb)=0$$
	 $$g(a)g(t)-g(s)g(b)=0$$	 
	 $$g(a)g(t)=g(s)g(b)$$
	 $$g(s)^{-1}g(a)g(t)g(t)^{-1}=g(s)^{-1}g(s)g(b)g(t)^{-1}$$
	 $$g(s)^{-1}g(a)=g(b)g(t)^{-1}$$
	 
	 Como estamos trabalhando com anéis comutativos
	 
	 $$h(a,s)=g(a)g(s)^{-1}=g(b)g(t)^{-1}=h(b,t)$$
	 
	 Ou seja, $h$ não depende dos representantes escolhidos.
	 
	 Agora vamos mostrar que $h$ é um homomorfismo.
	 
	 \begin{enumerate}
	 	\item[soma] Queremos mostrar que $h((a,s)+(b,t))=h(a,s)+h(b,t)$ 
	 	Sabemos que
	 	$$h((a,s)+(b,t))=h(at+sb,st)=g(at+sb)g(st)^{-1}$$
	 	$$g(at+sb)g(st)^{-1}=[g(a)g(t)+g(s)g(b)][g(s)g(t)]^{-1}$$
	 	$$[g(a)g(t)+g(s)g(b)]g(s)^{-1}g(t)^{-1}$$
	 	$$g(a)g(t)g(s)^{-1}g(t)^{-1}+g(s)g(b)g(s)^{-1}g(t)^{-1}$$
	 	Utilizando a comutatividade do anel $B$
	 	$$g(a)g(t)g(t)^{-1}g(s)^{-1}+g(b)g(s)g(s)^{-1}g(t)^{-1}$$
	 	$$g(a)g(s)^{-1}+g(b)g(t)^{-1}=h(a,s)+h(b,t)$$
	 	\item[produto] Queremos mostrar que $h((a,s)(b,t))=h(a,s)h(b,t)$
	 	Sabemos que
	 	$$h((a,s)(b,t))=h(ab,st)$$
	 	$$h(ab,st)=g(ab)g(st)^{-1}$$
	 	$$g(ab)g(st)^{-1}=g(ab)[g(st)]^{-1}$$
	 	$$g(a)g(b)g(s)^{-1}g(t)^{-1}$$
	 	Como o anel $B$ é comutativo, temos que
	 	$$g(a)g(s)^{-1}g(b)g(t)^{-1}=h(a,s)h(b,t)$$	 	
	 \end{enumerate}
 
 	Portanto, temos que $h$ é homomorfismo.
 
 	Agora, basta mostrar que $g = h\circ f$.
 	
 	Temos que $$ h\circ f=h(f(a))=h(as,s)$$
 	$$h(as,s)=g(as)g(s)^{-1}=g(a)g(s)g(s)^{-1}=g(a)$$
 	
 	Portanto a composição se verifica.
	 
\end{proof}

\section{Módulo}

\begin{definition}
	Seja $A$ um anel associativo, comutativo e não necessariamente com unidade. Chamamos um conjunto $M$ não vazio de $A$-Modulo á esquerda se $M$ é um grupo abeliano com uma operação que vamos denotar por + e se está definida uma lei de composição externa que a cada par $(\alpha, m) \in A\times M$ associa a um elemento $\alpha m \in M$ e tal que para todos $\alpha_1, \alpha_2 \in A$ e $m_1, m_2 \in M$, verifica que: 
	\begin{enumerate}
		\item $\alpha_1(\alpha_2 m_1) = (\alpha_1 \alpha_2)m_1$
		\item $\alpha_1(m_1+m_2) = \alpha_1 m_1 + \alpha_1 m_2$
		\item $(\alpha_1+\alpha_2)m_1=\alpha_1 m_1+\alpha_2 m_1$
	\end{enumerate}
\end{definition}

\begin{remark}
	Os módulos podem ser definidos para anéis com unidades, mas para isso precisamos adicionar a condição $1m_1=m_1$ para $m_1 \in M$. Um módulo com essa propriedade é chamado de módulo unital.
\end{remark}

\begin{example}
	Todo espaço vetorial sobre um corpo K é um K-Módulo.
\end{example}

\begin{example}
	Se tomarmos um ideal $I$ de um anel $A$, $I$ é um $A$-Modulo
\end{example}

\begin{example}
	Se tomarmos um grupo abeliano $g$, com a seguinte operação $nx = x+x+...+x$ onde $n \in \mathbb{Z}$ e $x \in g$, temos que $g$ é um $\mathbb{Z}$-Modulo.
\end{example}

\begin{definition}
	Seja M um A-modulo. Um subconjunto $N \subseteq M$ é dito um A-submódulo de M se:
	\begin{enumerate}
		\item N é um subgrupo aditivo de M
		\item Para todo $\alpha \in A$ e $n \in N$, temos que $an \in N$
	\end{enumerate}
\end{definition}

\begin{proposition}
	Seja M, um A-módulo, então vale as seguintes propriedades:
	\begin{enumerate}
		\item $0m=0$ para todo $m \in M$
		\item $(-a)m=a(-m)=-(am)$ para todo $a \in A$ e $m \in M$
	\end{enumerate}
\end{proposition}

\begin{proof}
	\begin{enumerate}
		\item $0m=0$, de fato, pois $0m_1=(0+0)m_1=0m_1+0m_1$ adicionando $-0m_1$ em ambos lados ficamos com $0m_1=0$
		\item $(-a)m=a(-m)=-(am)$ para $m \in M$ e $a \in A$, de fato, pois $(-a)m=(-a)m+am-(am)=(-a+a)m-(am)=-(am)$ e de forma análoga $a(-m)=a(-m)+am-(am)=a(-m+m)-(am)=-(am)$
	\end{enumerate}
\end{proof}

\begin{proposition}
	Um subconjunto n não vazio de m é um submódulo se e somente se
	\begin{enumerate}
		\item Para todo $n_1, n_2 \in n$ temos que $n_1+n_2 \in n$
		\item Para todo $a \in A$ e $n \in N$ temos que $an \in N$
	\end{enumerate}
\end{proposition}
\begin{proof}
	Para a ida, temos que se N é um submódulo, então n é um subgrupo aditivo de M, logo para $n_1, n_2 \in n$ temos que $n_1+n_2$. A segunda condição da proposição é exatamente igual a segunda condição de submódulo.
	Para a volta, temos que  $n_1, n_2 \in n$ temos que $n_1+n_2 \in n$, isto é, n é fechado na soma. Sabemos pelo segundo item que a multiplicação de um elemento do subconjunto com um elemento do anel deve estar no subconjunto, em particular $0 n_1 = 0 \in n$(pelo item i da proposição anterior), então o elemento neutro está em n. Finalmente, sabemos que para todo $n_1$ implica em $-n_1 \in n$, pois $(-1)(n_1)=-n_1$, então $-n_1 \in n$(pelo item ii da proposição anterior). Logo n é subgrupo aditivo. E a segunda condição é idêntica.
\end{proof}

\begin{definition}
	Sejam $M$ e $N$ dois $A$-Módulos. Uma função $f:M \rightarrow N$ diz-se um homomorfismo de $A$-módulos se para todo $m_1, m_2 \in M$ e todo $a \in A$
	\begin{enumerate}
		\item $f(m_1+m_2)=f(m_1)+f(m_2)$
		\item $f(am_1)=af(m_1)$
	\end{enumerate}
\end{definition}

\begin{definition}
	Sejam $F, G, H$ três $A$-módulos e $f:F\rightarrow G$, $g:G\rightarrow H$ $A$-morfismos. Diz-se que o diagrama:
	$$F \xrightarrow{f} G \xrightarrow{g}
	H$$
	é uma sequência de ordem 2 em $G$ se $im(f)$subset$ ker(g)$.
	Em particular, se $im(f)=ker(g)$ o diagrama diz-se uma sequência exata em G.
\end{definition}

\begin{proposition}
	Seja $f:F \rightarrow G$ e $g:G \rightarrow H$ $A$-morfismos então
	$im(f)\subset ker(g)$ se e somente se $g \circ f =0$
\end{proposition}
\begin{proof}
	$\Rightarrow$Temos que se	
	$im(f)\subset ker(g)$, então $g(f(a))=0$ para todo $a \in F$, mas $g(f(a))=g \circ f(a)=0$
	
	$\Leftarrow$ $g \circ f =0$, então $f(a) \in ker(g)$ para todo $a \in F$, logo $im(f) \subset ker(g)$.
\end{proof}


\subsection{Localização em módulos}

Para fazer localização em módulos sobre anéis comutativos vamos seguir um caminho parecido.

\begin{definition}
	Seja A um anel associativo e comutativo, S um conjunto multiplicativo de A e M é um módulo sobre A. Vamos definir uma relação em $M \times S$ como $$(m_1,s_1) \equiv (m_2, s_2) \Leftrightarrow s(s_1m_2-s_2m_1)=0$$ para algum  $s \in S$.
\end{definition}

\begin{proposition}
	A relação definida anteriormente é uma relação de equivalência em $M \times S$.
\end{proposition}

\begin{proof} Vamos mostrar que é um relação reflexiva, simétrica e transitiva
	\begin{enumerate}
		\item [Reflexiva]  $(m_1,s_1) \equiv (m_1, s_1)$, temos que $s_1m_1-s_1m_1=0$, logo $s$ pode ser qualquer elemento de $S$.
		
		
		\item [Simétrica] Temos que $(m_1,s_1) \equiv (m_2, s_2)$ nos leva a $t(s_1m_2-s_2m_1)=0$ onde $s_im_j \in M$, como M é um grupo abeliano com a soma, temos que podemos os termos de lugar e pela propriedade que vimos na proposição anterior $-s(m)=s(-m)$, logo ao somar os opostos temos $$-t(s_1m_2-s_2m_1)+t(s_1m_2-s_2m_1)=-t(s_1m_2-s_2m_1)$$
		$$t(-s_1m_2+s_2m_1)=t(s_2m_1-s_1m_2)=0$$ logo temos que $(m_2,s_2) \equiv (m_1, s_1)$.
		
		
		\item [Transitividade] Seja $(m_1,s_1) \equiv (m_2,s_2)$ e $(m_2,s_2) \equiv (m_3,s_3)$, devemos chegar em $(m_1,s_1) \equiv (m_3,s_3)$.
		De $(m_1,s_1) \equiv (m_2,s_2)$ temos $t_1(s_1m_2-s_2m_1)=0$ para algum $t_1 \in S$.
		
		De $(m_2,s_2) \equiv (m_3,s_3)$ temos $t_2(s_2m_3-s_3m_2)=0$ para algum $t_2 \in S$.
		
		Multiplicando a primeira equação por $t_2s_3$ e a segunda por $t_1s_1$ chegamos a 
		$$t_2t_1s_3(s_1m_2-s_2m_1)=0 \ (*)$$
		$$t_1t_2s_1(s_2m_3-s_3m_2)=0 \ (**)$$
		
		Lembrando que os únicos elementos de $M$ são os $m_i$, o resto pertence a $S \subseteq A$, que é comutativo, logo podemos trocar a ordem dos elementos.
		
		Somando as equações (*) e (**) e realizando as distributivas temos 
		$$t_2t_1s_3s_1m_2-t_2t_1s_3s_2m_1+t_1t_2s_1s_2m_3-t_1t_2s_1s_3m_2=0$$
		
		Reorganizando os termos com a comutatividade temos
		
		$$t_1t_2s_1s_3m_2-t_1t_2s_2s_3m_1+t_1t_2s_1s_2m_3-t_1t_2s_1s_3m_2=0$$
		
		O primeiro e o último termo são idênticos a menos de um sinal, logo ficamos com 
		
		$$-t_1t_2s_2s_3m_1+t_1t_2s_1s_2m_3=0$$
		
		Como $M$ é um grupo abeliano, podemos trocar os termos de lugar e usando a distributiva do módulo temos 
		
		$$t_1t_2s_2(s_3m_1-s_1m_3)$$
		
		Como S é multiplicativo e $t_1, t_2 \in S$, logo $t_1t_2s_2 \in S$ e portanto $(m_1,s_1) \equiv (m_3, s_3)$.
	\end{enumerate}
	
	Portanto, a relação definida em $M \times S$ é de equivalência.
	
\end{proof}

\begin{notation}
	Denotamos por $S^{-1}M$ o conjunto das classes de equivalência da relação acima.
\end{notation}

Agora vamos mostrar que $S^{-1}M$ é um $S^{-1}A$-Módulo.

\begin{proposition}
	$S^{-1}M$ é um $S^{-1}A$-Módulo com as operações $$(m_1,s_1)+(m_2,s_2)=(s_1m_2+s_2m_1,s_1s_2)$$ e $$(a_1,s_3)*(m_1,s_1)=(a_1m_1,s_3s_1)$$.
\end{proposition}

\begin{proof}
	Devo mostrar que $S^{-1}M$ é um grupo abeliano com a soma e temos a operação de compatibilidade entre $S^{-1}A$ e $S^{-1}M$ bem definida. Mas antes disso precisamos mostrar que a soma está bem definida, ou seja, não depende dos representantes de classe.
	
	\begin{enumerate}
		\item A operação de soma está bem definida. Seja $(m_1,s_1)=(m_2,s_2)$ e tome $(m,s)$, queremos mostrar que $(m_1,s_1)+(m_s)=(m_2,s_2)+(m,s)$.
		Temos que $(m_1,s_1)+(m,s)=(s_1m+sm_1,s_1s)$ e $(m_2,s_2)+(m,s)=(s_2m+sm_2,s_2s)$. Da hipótese da igualdade das classes temos que $u(s_1m_2-s_2m_1)=0$ para algum $u \in S$.
		Queremos mostrar que
		$$(s_1m+sm_1,s_1s)=(s_2m+sm_2,s_2s)$$
		Sabemos que $$(s_2m+sm_2)s_1s-(s_1m+sm_1)s_2s$$
		Mas como estamos trabalhando com comutatividade podemos reorganizar e aplicando a distributiva, temos
		$$ss_1s_2m+sss_1m_2-ss_1s_2m-sss_2m_1=ss(s_1m2-s_2m1)$$
		Podemos multiplicar por $u$
		$$ssu(s_1m2-s_2m1)=ss0=0$$
		Portanto $(s_1m+sm_1,s_1s)=(s_2m+sm_2,s_2s)$
		
		\item $S^{-1}M$ é um grupo abeliano com a operação $(m_1,s_1)+(m_2,s_2)$=$(s_1m_2+s_2m_1,s_1s_2)$, onde no primeiro termo temos a soma no módulo e no segundo temos o produto do anel.
		
		\begin{enumerate}
			\item [Comutativa] $$(m_1,s_1)+(m_2,s_2) = (s_1m_2+s_2m_1,s_1s_2)$$
			$$(m_2,s_2)+(m_1,s_1) = (s_2m_1+s_1m_2,s_2s_1)$$
			
			Como a primeira coordenada é um elemento de M, onde M é um grupo abeliano, logo temos a comutatividade na primeira coordenada e na segunda coordenada temos elementos de um anel comutativo, logo a segunda coordenada também comuta e portanto as duas equações são iguais.
			
			\item [Associativa] $$((m_1,s_1)+(m_2,s_2))+(m_3,s_3)=(s_1m_2+s_2m_1,s_1s_2)+(m_3,s_3)$$
			$$(s_3(s_1m_2+s_2m_1)+s_1s_2m_3,s_1s_2s_3)=(s_3s_1m_2+s_3s_2m_1+s_1s_2m_3,s_1s_2s_3)$$
			
			Agora desenvolvendo por outra ordem
			$$(m_1,s_1)+((m_2,s_2)+(m_3,s_3))=(m_1,s_1)+(s_3m_2+s_2m_3,s_3s_2)$$
			$$=(s_1(s_3m_2+s_2m_3)+s_3s_2m_1,s_1s_3s_2)=(s_1s_3m_2+s_1s_2m_3+s_3a_2m_1, s_1s_3s_2)$$
			
			Pela comutatividade de cada coordenada temos que as duas equações são iguais.
			
			\item[Elemento neutro] $(0,s)$, temos que $(m_1,s_1)+(0,s)=(s_1 0+m_1s,s_1s)=(m_1s,s_1s)$
			
			Mas note que $(m_1s,s_1s)=(m_1,s_1)$, ou seja, são classes equivalentes. De fato, pois 
			
			$$m_1ss_1-m_1s_1s=0$$
			
			Pela comutatividade, logo podemos tomar qualquer $u \in S$ de tal forma que $u(m_1ss_1-m_1s_1s)=0$
			
			Portanto, $(m_1,s_1)+(0,s)=(m_1s,s_1s)=(m_1,s_1)$.
			
			\item[Elemento inverso] Para $(m_1,s_1)$ o elemento neutro seria $(-m_1,s_1)$
			
			$(m_1,s_1)+(-m_1,s_1)$ tem que ser igual a alguém da classe de $(0,s)$. 
			
			$$(m_1,s_1)+(-m_1,s_1)=(s_1m_1-s_1m_1, s_1s_1)=(0,s_1s_1)$$
			
			Então temos que mostrar que $(0,s_1s_1)$ é da mesma classe que $(0,s)$.
			
			$(0,s_1s_1)=(0,s)$ se e somente se existe $u \in S$ tal que $u(0s-0s_1s_2)=0$, mas como $(0s-0s_1s_2)=0$, logo $u$ pode ser qualquer elemento de $S$, portanto $(m_1,s_1)+(-m_1,s_1)=(0,s_1s_1)=(0,s)$
		\end{enumerate}
		
		Logo $S^{-1}M$ é um grupo abeliano.
		
		\item Operação compatibilidade entre $S^{-1}A$ e $S^{-1}M$
		$(a_1,s_3)*(m_1,s_1)=(a_1m_1,s_3s_1)$, onde $a_1\in A, m_1 \in M$ e $s_1, s_3 \in S$.
		
		Como $a_1m_1 \in M$, por $M$ ser um A-modulo. Como  $s_3 \in S \subseteq A$, logo $s_3s_1 \in A$, portanto está bem definida a operação.
		
	\end{enumerate}
\end{proof}

\begin{proposition}
	Seja $u:M \rightarrow N$ um $A-$homomorfismo. Então nós temos que $S^{-1}A$-modulo homomorfismo $S^{-1}u:S^{-1}M \rightarrow S^{-1}N$ que toma $(m,s)$ para $(u(m),s)$.
\end{proposition}
\begin{proof}
	Para mostrar que $S^{-1}u$ é um homomorfismo, devemos mostrar que 
	\begin{enumerate}
		\item $S^{-1}u(m,s)+S^{-1}u(n,t)=S^{-1}u((m,s)+(n,t))$
		
		$S^{-1}u(m,s)+S^{-1}u(n,t)=(u(m),s)+(u(n),t)=(tu(m)+su(n),st)=(u(tm)+u(sn),st)=
		(u(tm+sn),st)=S^{-1}u(tm+sn,st)=S^{-1}u((m,s)+(n,t))$
		
		\item $S^{-1}u((a,r)*(m,s))=(a,r)*S^{-1}u(m,s)$, onde $(a,r) \in S^{-1}A$
		
		$S^{-1}u((a,r)*(m,s))=S^{-1}u(am,rs)=(u(am),rs)=(au(m),rs)=(a,r)*(u(m),s)=(a,r)*S^{-1}u((m,s))$
	\end{enumerate}
\end{proof}

\begin{proposition}
	A operação $S^{-1}$ é exata, isto é, se $M' \xrightarrow{f} M \xrightarrow{g}  M''$ é exata em $M$, então $S^{-1}M' \xrightarrow{S^{-1}f} S^{-1}M \xrightarrow{S^{-1}g}  S^{-1}M''$ é exata em $S^{-1}M$.
\end{proposition}
\begin{proof}
	Sabemos que a sequência $M' \xrightarrow{f} M \xrightarrow{g}  M''$ é exata, portanto temos que $ker (g) = im (f)$, ou seja, todo mundo na imagem de f pertence ao núcleo de g, assim temos que $g(f(x))=0$ para todo x, portanto $g \circ f = 0$.
	
	Lembrando que se tivermos o homomorfismo $f:A \rightarrow B$, então o homomorfismo $S^{-1}f:S^{-1}A \rightarrow S^{-1}B$ é dado por $S^{-1}f(a,s)=(f(a),s)$.
	
	Para mostrar que a sequência $S^{-1}M' \xrightarrow{S^{-1}f} S^{-1}M \xrightarrow{S^{-1}g}  S^{-1}M''$ é exata, devemos mostrar que $ker (S^{-1}g) = im (S^{-1}f)$.
	
	Primeiro vamos mostrar que $im (S^{-1}f) \subset ker (S^{-1}g)$.
	
	Tome $(a,s)$ na imagem de $S^{-1}f$
	Temos que $S^-1g \circ S^-1f(a,s)=S^-1g(f(a),s)=(g(f(a)),s)$, mas como a sequência é exata, logo $g(f(x))=0$ e portanto $(g(f(x)),s)=(0,s)$.
	
	Agora vamos mostrar que 
	
	
\end{proof}

Itens a resolver nesse texto

\begin{enumerate}
	\item Arrumar a demonstração da proposição 2.17
	\item Mostrar que a função da proposição 2.16 está bem definida(classes de equivalência)
	\item Arrumar a demonstração da proposição 2.15, para 4 termos e não 3
	\item Proposição 1.25
	\item Proposição 1.15
\end{enumerate}




\newpage

\begin{thebibliography}{20}
	
	\bibitem{Atiyah}  Atiyah M. F.; MacDonald M. G., Introduction to Commutative Algebra . Addison-wesley publishing company, 1969.
	
	\bibitem{Fraleigh}  Fraleigh, J. B., A first course in abstract algebra. Person , 2003.
	
	\bibitem{Herstein} Herstein I. N., Topics in algebra. University of Chicago, 1975.	
	
\end{thebibliography}

\end{document}

